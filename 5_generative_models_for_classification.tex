\documentclass[10pt]{article}
\usepackage[utf8]{inputenc}
\usepackage[T1]{fontenc}
\usepackage{amsmath,amsfonts,amssymb}
\usepackage[version=4]{mhchem}
\usepackage{stmaryrd}
\usepackage{geometry}
\geometry{a4paper, margin=1in}

\title{Generative Models for Classification}
\author{}
\date{}

\begin{document}

\maketitle

\section{Introduction and Motivating Example}

\begin{itemize}
    \item Throughout this text, we will consider a simple yet illustrative example: \textit{classifying emails as spam or not spam}.
    \item In this example, we need a method to decide, based on features extracted from an email, whether it should be flagged as spam.
    \item There are two broad approaches to tackling this problem, distinguished by the type of probabilistic model used.
\end{itemize}

\subsection{Generative Models}
\begin{itemize}
    \item Generative models learn the joint probability distribution \(P(X,y)\) over the input features \(X\) and the labels \(y\).
    \item This allows generative models to not only classify emails using Bayes’ Theorem but also generate new emails that resemble the original data.
    \item Examples of generative models include the Bayes classifier, Linear Discriminant Analysis (LDA), Quadratic Discriminant Analysis (QDA), Variational Autoencoders (VAEs), and Generative Adversarial Networks (GANs).
\end{itemize}

\subsection{Discriminative Models}
\begin{itemize}
    \item Discriminative models directly model the conditional probability \(P(y \mid X)\), focusing solely on predicting the label from given features.
    \item These models often achieve high classification accuracy when ample data is available but do not attempt to model how the data is generated.
    \item Examples of discriminative models include K-Nearest Neighbors (KNN) and Logistic Regression.
\end{itemize}

\subsection{Spam Detection Example}
\begin{itemize}
    \item A generative model might learn how the words, sender information, or other features are jointly distributed in spam versus non-spam emails and then use Bayes' Theorem to classify.
    \item A discriminative model would learn a direct mapping from email features to the corresponding label, skipping the explicit modeling of how the email data is generated.
\end{itemize}

\section{Bayes Classifier}

The \textbf{Bayes classifier} provides the theoretical optimum under 0-1 loss by choosing the class that maximizes the posterior probability \(P(y \mid X)\). Specifically:
\[
p(y=k \mid X) \;=\; \frac{p(X \mid y=k)\,p(y=k)}{p(X)}.
\]
\begin{itemize}
  \item \(p(y = k \mid X)\) is the posterior probability: the probability of class \(k\) given features \(X\).
  \item \(p(X \mid y = k)\) is the class conditional distribution.
  \item \(p(y = k)\) is the prior probability of class \(k\).
  \item \(p(X)\) is the same for every class, so it is often ignored when comparing classes.
\end{itemize}

\noindent
\textbf{Practical Implementation:}
\begin{itemize}
  \item \textbf{In our spam example}, the prior \(p(\text{spam})\) might be estimated by the proportion of spam emails in your dataset, and \(p(X \mid \text{spam})\) might capture how likely it is to see particular words, senders, or other features in a spam email.
  \item The predicted class is the one with the highest posterior probability:
  \[
  \hat{y} = \arg\max_{k} \; p(y = k \mid X).
  \]
  \item Under 0-1 loss, this is the \emph{Bayes decision rule}. It yields the lowest achievable classification error if the true distributions are known.
\end{itemize}

\section{Why Not Always Use the Bayes Classifier?}

Although the Bayes classifier is theoretically optimal under 0-1 loss, we typically do not know the true class conditional probabilities \(p(X \mid y=k)\) or the priors \(p(y=k)\). We must \emph{estimate} them from the data, which can be challenging:
\begin{itemize}
  \item If we make wrong assumptions (e.g., assuming a distribution is normal when it is not), the classifier might perform poorly.
  \item If we have limited data, accurately estimating high-dimensional joint distributions can be difficult.
  \item Discriminative approaches often yield better performance in many practical scenarios because they bypass modeling the full joint distribution.
\end{itemize}

\section{Parameter Estimation: Priors and Conditionals}

\subsection{Estimating Class Priors}

A straightforward way to estimate the prior for each class (e.g., spam or not spam) is to count how many training samples belong to that class:
\[
\pi_k = \frac{n_k}{n},
\]
where \(n_k\) is the number of training samples of class \(k\) and \(n\) is the total number of training samples.

\subsection{Estimating Class Conditionals}

\paragraph{Histogram Approximation (Naive Bayes Example)} 
Although less common in advanced exams, a simple way to estimate \(p(X \mid y = k)\) is to:
\begin{itemize}
  \item Split the data by class.
  \item Form a histogram (or multiple histograms if \(X\) has many dimensions) to approximate the frequency distribution.
  \item If we assume features are conditionally independent given the class, we multiply the estimated probabilities for each feature dimension to obtain \(p(X \mid y = k)\).
\end{itemize}

\noindent
\textbf{Spam Example:} For each word (treated as a feature) within a spam email, we could estimate a histogram of word occurrences in spam emails vs. non-spam emails. Under a naive assumption of independence, we combine these estimates to compute the probability of seeing a particular combination of words in a spam email. Then we multiply by the prior \(p(\text{spam})\).

\section{Linear and Quadratic Discriminant Analysis (LDA and QDA)}

\begin{itemize}
    \item \textbf{Context: Generative Models for Classification}
    \begin{itemize}
        \item Generative models model the joint distribution \(P(X,y)\) to both classify and generate data.
        \item LDA and QDA are practical implementations based on Bayes classifiers that estimate posterior probabilities to minimize misclassification.
    \end{itemize}
    
    \item \textbf{Connection to Bayes Classifier}
    \begin{itemize}
        \item Both LDA and QDA derive from Bayesian decision theory.
        \item They estimate \(P(y \mid X)\) by modeling class-conditional densities and class priors.
    \end{itemize}
    
    \item \textbf{Multivariate Gaussian Assumption}
    \begin{itemize}
        \item Features \(x \in \mathbb{R}^p\) are assumed to follow a multivariate normal distribution.
        \item Density function:
        \[
        p(x; \mu, \Sigma) = \frac{1}{\underbrace{(2\pi)^{p/2}}_{\substack{\text{Normalization for} \\ \text{\(p\) dimensions}}} \underbrace{|\Sigma|^{1/2}}_{\substack{\text{Scaling by} \\ \text{covariance determinant}}}} \exp\!\Bigl(-\frac{1}{2}(x-\mu)^\top\Sigma^{-1}(x-\mu)\Bigr)
        \]
        \item \((x-\mu)^\top \Sigma^{-1}(x-\mu)\) represents the Mahalanobis distance between \(x\) and the mean \(\mu\).
    \end{itemize}
    
    \item \textbf{Linear Discriminant Analysis (LDA)}
    \begin{itemize}
        \item \textbf{Assumptions:}
        \begin{itemize}
            \item All classes share a common covariance matrix \(\Sigma\).
            \item Each class \(C_k\) has its own mean vector \(\mu_k\).
        \end{itemize}
        \item \textbf{Class-Conditional Density:} For a class \(C_k\),
        \begin{itemize}
            \item Formula:
            \[
            p(x \mid C_k) = \frac{1}{\underbrace{(2\pi)^{p/2}}_{\substack{\text{Normalization for} \\ \text{\(p\) dimensions}}} \underbrace{|\Sigma|^{1/2}}_{\substack{\text{Common scaling} \\ \text{across classes}}}} \exp\!\Bigl(-\frac{1}{2}(x-\mu_k)^\top\Sigma^{-1}(x-\mu_k)\Bigr)
            \]
            \item \(\mu_k\) is the mean of class \(C_k\) and \(\Sigma\) is shared across all classes.
        \end{itemize}
        \item \textbf{Discriminant Function:}
        \begin{itemize}
            \item Formula:
            \[
            \delta_k(x) = x^\top\Sigma^{-1}\mu_k - \frac{1}{2}\mu_k^\top\Sigma^{-1}\mu_k + \ln(\pi_k)
            \]
            \item Explanation:
            \begin{itemize}
                \item \(x^\top\Sigma^{-1}\mu_k\): weights the features \(x\) with the class mean \(\mu_k\) and covariance \(\Sigma\).
                \item \(-\frac{1}{2}\mu_k^\top\Sigma^{-1}\mu_k\): adjusts for the class-specific constant.
                \item \(\ln(\pi_k)\): incorporates the prior probability \(\pi_k\) of class \(C_k\).
            \end{itemize}
            \item \textbf{Decision Rule:} Assign \(x\) to the class with the highest \(\delta_k(x)\), producing linear decision boundaries.
        \end{itemize}
        \item \textbf{Example in Spam Detection:}
        \begin{itemize}
            \item Assume features such as total word count and number of suspicious keywords in an email follow a Gaussian distribution.
            \item LDA estimates the shared covariance and class means to classify emails as spam or not spam.
        \end{itemize}
    \end{itemize}
    
    \item \textbf{Quadratic Discriminant Analysis (QDA)}
    \begin{itemize}
        \item \textbf{Assumptions:}
        \begin{itemize}
            \item Each class \(C_k\) has its own covariance matrix \(\Sigma_k\).
            \item Each class \(C_k\) has its own mean vector \(\mu_k\).
        \end{itemize}
        \item \textbf{Class-Conditional Density:} For a class \(C_k\),
        \begin{itemize}
            \item Formula:
            \[
            p(x \mid C_k) = \frac{1}{\underbrace{(2\pi)^{p/2}}_{\substack{\text{Normalization for} \\ \text{\(p\) dimensions}}} \underbrace{|\Sigma_k|^{1/2}}_{\substack{\text{Scaling by class-} \\ \text{specific covariance}}}} \exp\!\Bigl(-\frac{1}{2}(x-\mu_k)^\top\Sigma_k^{-1}(x-\mu_k)\Bigr)
            \]
            \item \(\Sigma_k\) is specific to class \(C_k\) and allows for different covariance structures.
        \end{itemize}
        \item \textbf{Discriminant Function:}
        \begin{itemize}
            \item Formula:
            \[
            \delta_k(x) = -\frac{1}{2}\ln|\Sigma_k| - \frac{1}{2}(x-\mu_k)^\top\Sigma_k^{-1}(x-\mu_k) + \ln(\pi_k)
            \]
            \item Explanation:
            \begin{itemize}
                \item \(-\frac{1}{2}\ln|\Sigma_k|\): corrects for the class-specific scaling of the density.
                \item \(-\frac{1}{2}(x-\mu_k)^\top\Sigma_k^{-1}(x-\mu_k)\): measures the Mahalanobis distance in the context of \(\Sigma_k\).
                \item \(\ln(\pi_k)\): includes the prior probability for class \(C_k\).
            \end{itemize}
            \item \textbf{Decision Rule:} Assign \(x\) to the class with the highest \(\delta_k(x)\), leading to quadratic decision boundaries.
        \end{itemize}
        \item \textbf{Example in Spam Detection:}
        \begin{itemize}
            \item In cases where variability among features differs between spam and non-spam emails, QDA allows different covariance matrices for each class.
            \item This flexibility can improve classification when the Gaussian assumption holds differently in each class.
        \end{itemize}
    \end{itemize}
    
    \item \textbf{Comparison of LDA vs QDA}
    \begin{itemize}
        \item \textbf{Covariance Structure:}
        \begin{itemize}
            \item LDA: Uses a common covariance matrix \(\Sigma\) for all classes.
            \item QDA: Uses class-specific covariance matrices \(\Sigma_k\).
        \end{itemize}
        \item \textbf{Decision Boundaries:}
        \begin{itemize}
            \item LDA: Results in linear boundaries.
            \item QDA: Results in quadratic boundaries.
        \end{itemize}
        \item \textbf{Parameter Estimation:}
        \begin{itemize}
            \item LDA: Fewer parameters to estimate, leading to lower variance.
            \item QDA: More parameters, offering flexibility at the risk of higher variance.
        \end{itemize}
        \item \textbf{Data Requirements:}
        \begin{itemize}
            \item LDA: Often preferred when data is limited.
            \item QDA: Requires larger datasets for reliable estimation of class-specific covariances.
        \end{itemize}
        \item \textbf{Bias-Variance Tradeoff:}
        \begin{itemize}
            \item LDA: Typically higher bias but lower variance.
            \item QDA: Typically lower bias but higher variance.
        \end{itemize}
    \end{itemize}
    
    \item \textbf{Parameter Estimation for LDA and QDA (Generative Framework)}
    \begin{itemize}
        \item \textbf{Class Priors:}
        \begin{itemize}
            \item Estimated as:
            \[
            \pi_k = \frac{n_k}{n}
            \]
            \item \(n_k\): Number of samples in class \(C_k\); \(n\): Total number of samples.
        \end{itemize}
        \item \textbf{Class Means:}
        \begin{itemize}
            \item Estimated as:
            \[
            \hat{\mu}_k = \frac{1}{n_k}\sum_{i:\,y_i=k}\boldsymbol{x}_i
            \]
            \item \(\hat{\mu}_k\): Sample mean vector for class \(C_k\).
        \end{itemize}
        \item \textbf{Covariance Estimation for LDA:}
        \begin{itemize}
            \item \textbf{Single Feature:}
            \begin{itemize}
                \item Estimated variance:
                \[
                \hat{\sigma}^2 = \frac{1}{n-K}\sum_{k=1}^K\sum_{i:\,y_i=k}\underbrace{(x_i-\hat{\mu}_k)^2}_{\substack{\text{Squared deviation} \\ \text{from mean}}}
                \]
            \end{itemize}
            \item \textbf{Multiple Features:}
            \begin{itemize}
                \item Estimated covariance matrix:
                \[
                \hat{\Sigma} = \frac{1}{n-K}\sum_{k=1}^K\sum_{i:\,y_i=k}\underbrace{(\boldsymbol{x}_i-\hat{\mu}_k)(\boldsymbol{x}_i-\hat{\mu}_k)^\top}_{\substack{\text{Outer product of} \\ \text{deviations from mean}}}
                \]
            \end{itemize}
        \end{itemize}
        \item \textbf{Covariance Estimation for QDA:}
        \begin{itemize}
            \item \textbf{Single Feature:}
            \begin{itemize}
                \item Estimated class-specific variance:
                \[
                \hat{\sigma}_k^2 = \frac{1}{n_k-1}\sum_{i:\,y_i=k}\underbrace{(x_i-\hat{\mu}_k)^2}_{\substack{\text{Squared deviation} \\ \text{from mean}}}
                \]
            \end{itemize}
            \item \textbf{Multiple Features:}
            \begin{itemize}
                \item Estimated class-specific covariance matrix:
                \[
                \hat{\Sigma}_k = \frac{1}{n_k-1}\sum_{i:\,y_i=k}\underbrace{(\boldsymbol{x}_i-\hat{\mu}_k)(\boldsymbol{x}_i-\hat{\mu}_k)^\top}_{\substack{\text{Outer product of} \\ \text{deviations from mean}}}
                \]
            \end{itemize}
        \end{itemize}
        \item \textbf{Spam Detection Context:}
        \begin{itemize}
            \item LDA assumes a single covariance \(\Sigma\) for both spam and non-spam emails.
            \item QDA allows different covariance estimates (\(\Sigma_{\text{spam}}\) and \(\Sigma_{\text{non-spam}}\)) to capture varying feature variability.
        \end{itemize}
    \end{itemize}
    
    \item \textbf{Summary of LDA and QDA in a Generative Framework}
    \begin{itemize}
        \item They implement the Bayes classifier by modeling Gaussian class-conditional densities.
        \item LDA offers simplicity with linear decision boundaries and fewer parameters.
        \item QDA provides greater flexibility with quadratic boundaries by estimating class-specific covariances.
        \item Both methods enable classification by explicitly modeling how data is generated, which is valuable in applications such as spam detection.
    \end{itemize}
\end{itemize}

\pagebreak
\section{Advanced Generative Models}

\begin{itemize}
    \item \textbf{Variational Autoencoders (VAEs)}
    \begin{itemize}
        \item \textbf{Latent Variable Model:} VAEs introduce a latent variable \(z\) and learn both an encoder and a decoder.
        \begin{itemize}
            \item Encoder: \(q_\phi(z \mid x)\) \quad \(\underbrace{\text{Approximates the true posterior of } z}_{\text{latent representation estimation}}\)
            \item Decoder: \(p_\theta(x \mid z)\) \quad \(\underbrace{\text{Generates data from } z}_{\text{reconstruction of } x}\)
        \end{itemize}
        \item \textbf{Training Objective (ELBO):}
        \begin{itemize}
            \item \[
              \mathcal{L} = \mathbb{E}_{q_{\phi}(z \mid x)}\Bigl[\ln p_{\theta}(x \mid z)\Bigr] - D_{\text{KL}}\Bigl(q_{\phi}(z \mid x) \,\|\, p(z)\Bigr)
              \]
            \item \(\underbrace{\mathbb{E}_{q_{\phi}(z \mid x)}\big[\ln p_{\theta}(x \mid z)\big]}_{\substack{\text{Reconstruction term:}\\\text{accuracy of } x \text{ recovery}}}\)
            \item \(\underbrace{D_{\text{KL}}\bigl(q_{\phi}(z \mid x)\,\|\,p(z)\bigr)}_{\substack{\text{Regularization:}\\\text{closeness to prior } p(z)}}\)
        \end{itemize}
        \item \textbf{Data Generation:} VAEs generate new samples by drawing \(z\) from the prior \(p(z)\) and computing \( \hat{x} \sim p_\theta(x \mid z)\).
        \item \textbf{Classification Adaptation:} VAEs can be extended to learn class-dependent latent spaces for classification tasks.
        \item \textbf{Spam Example:} A VAE trained on emails learns a latent space where differences between spam and non-spam emails can be observed; this latent representation can be used for data augmentation or anomaly detection.
    \end{itemize}
    
    \item \textbf{Generative Adversarial Networks (GANs)}
    \begin{itemize}
        \item \textbf{Adversarial Framework:} GANs consist of two competing networks: a generator and a discriminator.
        \begin{itemize}
            \item Generator \(G\): Produces synthetic samples, \(x_{\text{fake}} = G(z)\), by mapping noise \(z\) (drawn from a simple prior, e.g., \(p(z)\)) to data space.
            \item Discriminator \(D\): Evaluates samples and distinguishes between real data \(x\) and generated data \(x_{\text{fake}}\); outputs a probability.
        \end{itemize}
        \item \textbf{Training Objective:}
        \begin{itemize}
            \item Generator Objective:
            \[
            \min_G \; \underbrace{\mathbb{E}_{z\sim p(z)}\Bigl[\ln\bigl(1-D(G(z))\bigr)\Bigr]}_{\substack{\text{Fool the discriminator}}}
            \]
            \item Discriminator Objective:
            \[
            \max_D \; \underbrace{\mathbb{E}_{x\sim p_{\text{data}}(x)}\Bigl[\ln D(x)\Bigr]}_{\substack{\text{Recognize real data}}} + \underbrace{\mathbb{E}_{z\sim p(z)}\Bigl[\ln\bigl(1-D(G(z))\bigr)\Bigr]}_{\substack{\text{Reject fake data}}}
            \]
        \end{itemize}
        \item \textbf{Data Augmentation for Classification:} Although GANs are not primarily designed for classification, they can augment datasets by generating additional samples (e.g., extra spam emails if they are underrepresented).
        \item \textbf{Semi-Supervised Learning:} The discriminator in GANs can be adapted to also classify data when only limited labeled examples are available.
        \item \textbf{Spam Example:} In a scenario with few spam emails, a GAN can generate realistic synthetic spam samples to enlarge the training set for a spam classifier.
    \end{itemize}
\end{itemize}

\end{document}
